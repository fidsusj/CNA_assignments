\section*{Problem 7-2 Molloy-Reed Criterion}

Consider a configuration model network that has nodes of degree 1, 2, and 3 only, with probabilities $p_1$, $p_2$, and $p_3$, respectively. The degree distribution is given by:

\begin{equation} \label{eq}
	p_k = \delta_{k,1} p_1 + \delta_{k,2} p_2 + \delta_{k,3} p_3, 
	\begin{cases}
		\delta_{k,1} = 3 & \text{if $k = 1$} \\
		\delta_{k,2} = 2 & \text{if $k = 2$} \\
		\delta_{k,3} = 1 & \text{if $k = 3$}
	\end{cases}
\end{equation}

\begin{enumerate}
	\item Compute the first moment $\langle k \rangle$ and the second moment $\langle k^2 \rangle$ of the degree distribution.
	
	We assume $\delta_{k,k'}$ to be the dirac-delta-function. For the first and second moment is follows:
	
	\begin{equation*}
		\langle k \rangle = \sum_{k = 1}^3 k p_k = 1p_1 + 2p_2 + 3p_3 = 3p_1 + 4p_2 + 3p_3
	\end{equation*}

	\begin{equation*}
		\langle k^2 \rangle = \sum_{k = 1}^3 k^2 p_k = 1p_1 + 4p_2 + 9p_3 = 3p_1 + 8p_2 + 9p_3
	\end{equation*}
	
	Note that we substitute $p_1$ with $3p_1$ and $p_2$ with $2p_2$ ($p_3$ remains $2p_3$) as given by equation \ref{eq} (slightly confusing by the task description).
	
	\item Using the Molloy-Reed criterion, show that there is a giant component if and only if $p_1 < 3p_3$.
	
	The Molloy-Reed criteria propagates a giant component exists in case $\kappa = \frac{\langle k^2 \rangle}{\langle k \rangle} > 2$. $\kappa$ can be calculated as:
	
	\begin{equation*}
		\kappa = \frac{\langle k^2 \rangle}{\langle k \rangle} = \frac{1p_1 + 4p_2 + 9p_3}{1p_1 + 2p_2 + 3p_3}
	\end{equation*}

	which is only true for 
	
	\begin{equation*}
		\begin{split}
			\frac{1p_1 + 4p_2 + 9p_3}{1p_1 + 2p_2 + 3p_3} > 2 \\
			1p_1 + 4p_2 + 9p_3 > 2p_1 + 4p_2 + 6p_3 \\
			3p_3 > p_1 \\
		\end{split}
	\end{equation*}

	Note that we use the LHS declaration of $p_k$ from equation \ref{eq}.
	
	\item In terms of the structure of the network, discuss the meaning of the condition $p_1 < 3p_3$. Why does the result not depend on $p_2$?
	
	For the network to have a giant component, the probability of a node having a single degree should be at most three times as high as the probability of a node having a degree of three. This limits the amount of single degree nodes and promotes a faster growth of a giant component since the average degree will most likely not be close to $\langle k \rangle = 1$ , but rather higher (assuming we exclude isolated nodes as in equation \ref{eq}) since single degree nodes cannot prevail the network. 
	
	The probability $p_2$ fell apart from the equation shown in subtask 2, leading to the assumption that the emergence of a giant component does not need to be dependent on $p_2$. This makes sense since we know the constraint $p_1 < 3p_3$ holds, which already leads to the corollar that $\langle k \rangle \ge 1$ and therefore leads to the guaranteed emergence of a giant component.
\end{enumerate}