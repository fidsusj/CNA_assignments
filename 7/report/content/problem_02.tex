\section*{Problem 7-2 Molloy-Reed Criterion}

Consider a configuration model network that has nodes of degree 1, 2, and 3 only, with probabilities $p_1$, $p_2$, and $p_3$, respectively. The degree distribution is given by:

\begin{equation*}
	p_k = \delta_{k,1} p_1 + \delta_{k,2} p_2 + \delta_{k,3} p_3, 
	\begin{cases}
		\delta_{k,1} = 3 & \text{if $k = 1$} \\
		\delta_{k,2} = 2 & \text{if $k = 2$} \\
		\delta_{k,3} = 2 & \text{if $k = 3$}
	\end{cases}
\end{equation*}

\begin{enumerate}
	\item Compute the first moment $\langle k \rangle$ and the second moment $\langle k^2 \rangle$ of the degree distribution.
	
	We assume $\delta_{k,k'}$ to be the dirac-delta-function. For the first and second moment is follows:
	
	\begin{equation*}
			\langle k \rangle = \sum_{k = 1}^3 k p_k = 3p_1 + 4p_2 + 6p_3
	\end{equation*}

	\begin{equation*}
			\langle k^2 \rangle = \sum_{k = 1}^3 k^2 p_k = 3p_1 + 8p_2 + 18p_3
	\end{equation*}
	
	\item Using the Molloy-Reed criterion, show that there is a giant component if and only if $p_1 < 3p_3$.
	
	The Molloy-Reed criteria propagates a giant component exists in case $\kappa = \frac{\langle k^2 \rangle}{\langle k \rangle} > 2$. $\kappa$ can be calculated as:
	
	\begin{equation*}
		\begin{split}
			\kappa = \frac{\langle k^2 \rangle}{\langle k \rangle} = \frac{3p_1 + 8p_2 + 18p_3}{3p_1 + 4p_2 + 6p_3}
		\end{split}
	\end{equation*}

	One way for $\kappa$ being greater than 2 is when all components of the sum in the numerator are at least twice as big as the matching component of the sum in the denominator (with one component being slightly bigger than twice as big for $\kappa$ being strictly bigger than 2). This means $6p_3 > 3p_1$ and therefore $2p_3 > p_1$. We therefore assume the thesis in the task desciption is wrong. 
	
	\item In terms of the structure of the network, discuss the meaning of the condition $p_3 < 3p_1$. Why does the result not depend on $p_2$?
	
	The fact that in the calculation of $\kappa$ the component of $p_2$ already goes in twice as big in the numerator ($8p_2 > 4p_2$), makes the goal of $\kappa > 2$ only dependent on the scale of $p_1$ compared to $p_3$. If $p_3 < 3p_1$ holds, a giant component only exists in case $\frac{1}{3}p_3 < p_1 < 2p_3$. Other than that, we cannot really see what $p_3 < 3p_1$ should lead to.
	
\end{enumerate}