% This template was initially provided by Dulip Withanage.
% Modifications for the database systems research group
% were made by Conny Junghans,  Jannik Strötgen and Michael Gertz

\documentclass[
     12pt,         % font size
     a4paper,      % paper format
     BCOR10mm,     % binding correction
     DIV14,        % stripe size for margin calculation
     ]{article}

%%%%%%%%%%%%%%%%%%%%%%%%%%%%%%%%%%%%%%%%%%%%%%%%%%%%%%%%%%%%

% PACKAGES:

% Use German
\usepackage[english]{babel}
% Input and font encoding
\usepackage[latin1]{inputenc}
\usepackage[T1]{fontenc}
% Index-generation
\usepackage{makeidx}
% Embedding of URLs
\usepackage{url}
% Special \LaTex symbols (e.g. \BibTeX)
%\usepackage{doc}
% Include Graphic-files
\usepackage{graphicx}
% Include doc++ generated tex-files
%\usepackage{docxx}
% Fuer anderthalbzeiligen Textsatz
\usepackage{setspace}
\usepackage[table,xcdraw]{xcolor}
\usepackage{hhline}
\usepackage{highlight}
\usepackage{amsmath}
\usepackage{subcaption}

\usepackage{biblatex}
\bibliography{references.bib}

% hyperrefs in the documents
\PassOptionsToPackage{hyphens}{url}\usepackage[bookmarks=true,colorlinks,pdfpagelabels,pdfstartview = FitH,bookmarksopen = true,bookmarksnumbered = true,linkcolor = black,plainpages = false,hypertexnames = false,citecolor = black,urlcolor=black]{hyperref}
%\usepackage{hyperref}


% CUSTOM:

% For quotes:
\usepackage{csquotes}

% For Definitions:                   						
\usepackage{amsthm}
\usepackage[framemethod=tikz]{mdframed}

% For graphs
\usepackage{tikz} 

% For PDFs
\usepackage{pdfpages}

% For math equations
\usepackage{amsmath}

\newtheoremstyle{defi}
{\topsep}         % Abstand oben
{\topsep}         % Abstand unten
{\normalfont}     % Schrift des Bodys
{0pt}             % Einschub der ersten Zeile
{\bfseries}       % Darstellung von der Schrift in der �berschrift
{:}               % Trennzeichen zwischen �berschrift und Body
{.5em}            % Abstand nach dem Trennzeichen zum Body Text
{\thmname{#3}}    % Name in eckigen Klammern
\theoremstyle{defi}

\newmdtheoremenv[
hidealllines = true,       % Rahmen komplett ausblenden
leftline = true,           % Linie links einschalten
innertopmargin = 0pt,      % Abstand oben
innerbottommargin = 4pt,   % Abstand unten
innerrightmargin = 0pt,    % Abstand rechts
linewidth = 3pt,           % Linienbreite
linecolor = gray!40,       % Linienfarbe
]{defStrich}{Definition}     % Name der des formats "defStrich"



%%%%%%%%%%%%%%%%%%%%%%%%%%%%%%%%%%%%%%%%%%%%%%%%%%%%%%%%%%%%

% OTHER SETTINGS:

% Choose language
\newcommand{\setlang}[1]{\selectlanguage{#1}\nonfrenchspacing}


\begin{document}

% TITLE:
\pagenumbering{roman} 
\begin{titlepage}

\vspace*{1cm}
\begin{center}
\textbf{ 
\Large Heidelberg University\\
\smallskip
\Large Institute of Computer Science\\
\smallskip
\Large Database Systems Research Group\\
\smallskip
}

\vspace{3cm}

\textbf{\large Lecture: Complex Network Analysis}

\vspace{0.5cm}
Prof. Dr. Michael Gertz

\vspace{2cm}

\vspace{0.5\baselineskip}
{\huge
\textbf{Assignment 7}
}

\textbf{\large Degree Assortativity and Robustness}
\vspace{0.5cm}

\url{https://github.com/nilskre/CNA_assignments}


\end{center}

\vfill 

{\large
\begin{tabular}[l]{ll}
Team Member: & Patrick G�nther, 3660886,\\
  & Applied Computer Science\\
  & rh269@stud.uni-heidelberg.de\\
  & \\
Team Member: & Felix Hausberger, 3661293,\\
  & Applied Computer Science\\
  & eb260@stud.uni-heidelberg.de\\
  & \\
Team Member: & Nils Krehl, 3664130,\\
  & Applied Computer Science\\
  & pu268@stud.uni-heidelberg.de\\
  
\end{tabular}
}

\end{titlepage}

\newpage

\pagenumbering{arabic}

%%%--------------------------------%%%
%%% Problem
%%%--------------------------------%%%


%%%--------------------------------%%%
%%% Problem N
%%%--------------------------------%%%
\foreach \i in {01,02,03,04,05,06,07,08,09,10,...,99} {%
	\edef\FileName{content/problem_\i}%
	\IfFileExists{\FileName}{%
		\input{\FileName}
	}
	{%
		% File does not exist
		
	}
}%foreach

\clearpage
\pagenumbering{Roman}

\end{document}
