\section{Problem 4-1 Power laws}

Consider the degree distribution functions $p_k$ of the following two undirected networks:

\begin{figure}[h]
	\centering
	\includegraphics[width=0.9\linewidth]{images/problem41_degree_distribution_networks.png}
	\caption{A simple graph with 4 connected triples.}
	\label{distribution}
\end{figure}

\begin{enumerate}
	\item One of these networks is approximately scale-free, the other is not. Identify the scale-free network and explain how you came to your conclusion.
	
	Network b is is scale free. 
	The figure \ref{distribution} shows the log-log plots of the degree distribution of the two networks a and b. For scale-free networks the degree distribution forms a straight line across the diagonal. This is given for network b.
	
	\item A particular network is believed to have a degree distribution that follows a power law. A random sample of nodes is taken and their degrees are measured. The degrees of the first twenty nodes with degrees 10 or greater are: 
	
	16  17  10  26  13  14  28  45  10  12
	12  10  136  16  25  36  12  14  22  10
	
	For $K_{min} = 10$, estimate the exponent $\gamma$ of the power law using the estimation method presented in the lecture.
	Hint: you only have to estimate the exponent (see Slide 4-33)!
	Calculate the error $\sigma$ of your estimation using the equation $sigma = \frac{\gamma - 1}{\sqrt{N}}$

\end{enumerate}