\section*{Problem 6-4 Degree Correlations in Random Graphs}

Consider the random model of undirected random networks $G(N,L)$, where $N$ nodes are connected with $L$ links that are placed uniformly at random. In this model, the probability $p(a_{ij} = 1)$ that an edge exists between nodes $i$ and $j$ is not independent from the existence of other edges since the overall number of edges $L$ in the graph is fixed. Note that this model is distinct from the $G(N,p)$ model in which each edge has an equal probability $p$ of existing.

\begin{enumerate}
	\item Give an equation for the probability $p(a_{ij} = 1)$ that the edge $(i,j)$ in the graph exists.
	
	By rearranging formula (3.3) from the lecture slides and solving for p, we obtain
	
	\begin{equation}
	p(a_{ij} = 1) = \frac{L}{\binom{N}{2}}
	\end{equation}
	
	\item Give equations for the probabilities that an other edge $(i,j)$ exists in the graph, conditioned on the existence of the edge $(x,y)$.
	
	If the edge $(x,y)$ does not exist in the graph $L$ edges are remaining to be distributed between the remaining $\binom{N}{2} - 1$ possibilities for edges to be placed between nodes.  If the edge $(x,y)$ does exist there are only $L-1$ edges remaining. Thus, the probabilities are
	
	\begin{equation}
	p(e_{ij}|e_{x,y}=0) = \frac{L}{\binom{N}{2} - 1},
	\end{equation}
	
	and
	
	\begin{equation}
	p(e_{ij}|e_{x,y}=1) = \frac{L - 1}{\binom{N}{2} - 1}.
	\end{equation}
	
	\item Derive the ratio of the conditional probabilities to the probability $p(a_{ij} = 1)$.
	
	Plugging in the formulas we got so far,  we get
	
	\begin{equation}
	r_0 =\frac{
					 \frac{L}{\binom{N}{2} - 1}
					 }
					 {
					 \frac{L}{\binom{N}{2}}
					 }
	\end{equation}
	
	\begin{equation}
	= \frac{\binom{N}{2}}{\binom{N}{2} - 1}
	\end{equation}
	
	\begin{equation}
	= \frac{N(N - 1)}{N(N - 1) - 2}
	\end{equation}
	
	\begin{equation}
	= \frac{N^2 - N}{N^2 - N - 2}
	\end{equation}
	
	and 
	
	\begin{equation}
	r_1 = \frac{
					 \frac{L - 1}{\binom{N}{2} - 1}
					 }
					 {
					 \frac{L}{\binom{N}{2}}
					 }
	\end{equation}
	
	\begin{equation}
	= \frac{
			L \cdot \binom{N}{2} - \binom{N}{2}
		}
		{
			L \cdot \binom{N}{2} - L		
		}
	\end{equation}
	
	\begin{equation}
	= \frac{
			\frac{
				LN(N - 1) - N(N - 1)
			}
			{
				2
			}		
		}
		{
			\frac{
				LN(N - 1) - 2L
			}
			{
				2
			}	
		}
	\end{equation}
	
	\begin{equation}
	= \frac{
			LN^2 - N^2 - LN + N
		}{
			LN^2 - LN - 2L
		}
	\end{equation}
	
	\item Give the ratios $r'_0$ and $r'_1$ for the $G(N,p)$ model.
	
	In the $G(N,p)$ model, the probability of every edge to be realized is determined by $p$. Edge probabilities are not dependent on other edges. This means
	
	\begin{equation}
		p(a_{ij} = 1) = p(e_{ij}|e_{x,y}=0) = p(e_{ij}|e_{x,y}=1),
	\end{equation}
	
	and thus
	
	\begin{equation}
		r'_0 = r'_1 = 1
	\end{equation}
	
	\item Discuss the implications of using the $G(N,L)$ instead of the $G(N,p)$ model. 
	
	For small networks, in the $G(N, p)$ model, the probability of an edge to exist is the same as in a network with many nodes. In the $G(N, L)$ model, the probability of an edge to exist is higher (if $L$ is kept the same) than in a large network mit many nodes.  This is because there are less possible realizations of $G$, if the number of nodes is small and the probability of an edge to exist is dependent on the existence of other edges in the $G(N,L)$ model.  If we look at the ratios $r_0$ and $r_1$, we can also see that for $L$ being the same, but $N$ getting small, $r_1$ decreases faster than $r_0$ ($-2L$ instead of $-2$ in the denominator, with the other terms decreasing for $N$ decreasing (does not go to zero)).
\end{enumerate}