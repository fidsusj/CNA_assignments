\section{Problem 6-1 The $t$-Party Evolving Network Model}

At the $t$-party gender plays no role, hence each newcome is allowed to invite \textbf{exactly} one other participant to a dance. However, attractiveness plays a role: More attractive participants are more likely to be invited to a dance
by a new participant. The party evolves following these rules:

\begin{itemize}
	\item Every participant corresponds to a node $i$ and is assigned a time-independent attractiveness coefficient $\eta_i$.
	\item At each time step a new node joins the $t$-party.
	\item This new node then invites one already partying node to a dance, establishing a new link with it.
	\item The new node chooses its dance partner with probability proportional to the potential partner's attractiveness. If there are $t$ nodes already at the party, the probability that node $i$ receives a dance invitation is:
	
	\begin{equation*}
		\Pi_i = \frac{\eta_i}{\sum_{j}^{} \eta_j} = \frac{\eta_i}{t \langle \eta \rangle}
	\end{equation*}

	where $\langle \eta \rangle$ is the average attractiveness.
\end{itemize}

\begin{enumerate}
	\item Derive the time evolution of the node degrees, telling us how many dances a node had.
	\item Derive the degree distribution of nodes with attractiveness $\eta$.
	\item If half of the nodes have $\eta=2$, and the other half $\eta=1$ , what is the degree distribution of the network after a sufficiently long time?
\end{enumerate}